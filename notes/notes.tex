\documentclass[a4paper,11pt]{article}
\usepackage{cmap}
\usepackage[T2A]{fontenc}
\usepackage{amssymb}
\usepackage[utf8]{inputenc}
\usepackage[unicode=true, colorlinks=true, linkcolor=black, urlcolor=blue, citecolor=black]{hyperref}
\usepackage[left=20mm, top=15mm, right=15mm, bottom=15mm]{geometry}
\usepackage[english,russian]{babel}
\usepackage{amsmath,amsfonts,mathtools,amssymb}
\usepackage{graphicx}
\author{Алексей Пеньков}
\title{Генерация музыки}
\date{}
\DeclareMathOperator{\ord}{ord}
\DeclareMathOperator{\im}{Im}
\DeclarePairedDelimiter\ceil{\lceil}{\rceil}
\DeclarePairedDelimiter\floor{\lfloor}{\rfloor}
\graphicspath{{pictures/}}
\DeclareGraphicsExtensions{.pdf,.png,.jpg}
\newcommand{\E}{\mathsf{E}}

\begin{document}
	\noindent
	\maketitle{}
	
	\section* {Компоненты музыки}
	В рамках предложенного алгоритма выделим мелодию и гармонию из всех средств музыкалой вы- разительности и будем рассматривать их генерацию на основе данного музыкального произведения.
	\section* {Выделение мелодии и гармонии}
	Первый шаг будет заключаться в выделении мелодии и гармонии из музыкального произведения. Тут и далее будем рассматривать представлние аудиофайлов в midi формате. Рассмотрим множе- ство всех нот $\Omega = \{\omega_{1} \ldots \omega_{127}\}$, и группы нот $W_{i} = \{\omega_{j} \ | \ j \equiv i \mod 12\}, i \in \{0 \ldots 11\}$. Отождествле- ние нот по модулю 12 разумно, так как октава состоит в точности из 12 нот.
	\subsection*{Аккорды}
	Теперь в каждый фиксированный момент $t$ музыку можно описать вектором $v_t$ размерности $12$. $i$-ая компонента вектора будет отвечать суммарной громкости группы нот $W_{i}$ в момент $t$. Вектор $v_t$ будем называть аккордом, соотвествующим времени $t$.
	\subsection*{Выделение последовательности аккордов}
	Процесс преобразования аудиофайла в последовательность аккордов будем называть выделенем мелодии и гармонии. Заметим также, что каждому аудиофайлу можно сопоставить конечную по- следовательность векторов $V = \{v_{t_0} \ldots v_{t_n}\}$, где моменты времени $t_0 \ldots t_n$ отвечают смене аккорда.
	\section*{Графы}
	Рассмотрим неоринтированный граф $ G = \{V, E\}, E = \Big\{(v_{t_i}, v_{t_{i+1}}) \ | \ i \in \{0 \ldots n - 1\}\Big\}$. В текущей
	версии вес ребра $(v_{t_i}, v_{t_{i+1}})$ определяется евклидовой нормой разности $v_{t_i} - v_{t_{i+1}}$. Попробуем решить задачу китайского почтальона на построенном графе. Задача заключается в поиске кратчайшего замкнутого пути или цикла, который проходит через каждое ребро связного взвешенного неориен- тированного графа. Если граф имеет эйлеров цикл, тогда этот цикл служит оптимальным решением. В противном случае задачей оптимизации является поиск подмножества рёбер с минимальным воз- можным общим весом, так что после из добавления получающийся мультиграф имел эйлеров цикл. Эта задача может быть решена за полиномиальное время.
	\section*{Сравнение результатов}
	Примечательно, что изначально построенная последовательность $\{v_{t_0} \ldots v_{t_n}, v_{t_0}\}$ также является решением задачи добавления ребер в граф $G$ для существования эйлерова цикла нем. Более того, такое решение оказывается близким к оптимальному на примере рассмотренных мной аудиофайлов.
	\section*{Cоздание новой версии мелодии и гармонии}
	После решения задачи китайского почтальона преобразуем найденный эйлеров цикл, представляю- щий собой последовательность аккордов, в midi файл \textbf{recreated-harmony.mid}. Это же сделаем с изначально построенной последовательностью $V$ , в результате получив midi файл \textbf{harmony.mid}.

\end{document}